\documentclass[a4paper]{article}

\usepackage[english]{babel}

\usepackage[a4paper,top=2cm,bottom=2cm,left=3cm,right=3cm,marginparwidth=1.75cm]{geometry}

\usepackage[%
  titleformat=italic,% Titles in italic 
  titleformat=commasep,% A comma between athors and title 
  titleformat=all,% Always show a title (or a short title)
  commabeforerest,% A comma after title 
  ibidem=strict,% 
  citefull=first,% The first citing in full form 
  oxford,% The oxford style
  super,% Footnotes 
  dotafter=true,% 
  see,% An extra optional argument as a prenote 
  idem
]{jurabib}

\usepackage{amsmath}
\usepackage{graphicx}
\usepackage[colorlinks=true, allcolors=blue]{hyperref}

\title{Elliptic curves in modern cryptography}
\author{Andrei Lupsa}

\begin{document}
\maketitle

\begin{abstract}
Abstract!!
\end{abstract}


\section{Introduction to elliptic curves}

An elliptic curve is an implicit function where one variable has a degree of 2 and the other has a degree of 3. 


\section{Elliptic curve point arithmetic}

\subsection{Elliptic curves as an abelian variety}

An abelian group is a kdfhjgkdfhjkg. A group $G$ that acts as an abelian group under an operation (such as addition, $+$) is written as $(G, +)$.

\subsection{Point addition \& doubling}

\subsubsection{Proof}

Take two points on an elliptic curve $E$ where $A(x_1, y_1) + B(x_2, y_2) = C(x_3, y_3)$ . Let the line through $A$, $B$ and $C'(x_3, -y_3)$ be $L$ such that
\begin{gather*}
    E: y^2 = x^2 + ax + b \\
    L: y = \lambda x + m \\
    \text{where } \lambda = \frac{y_2-y_1}{x_2-x_1}
\end{gather*}

The intersection of $E$ and $L$ satisfies the equation
\begin{align*}
    x^3 + ax + b = (\lambda x + m)^2 \\
    x^3 + ax + b = \lambda^2 x^2 + 2 \lambda x m + m^2 \\
    x^3 - \lambda^2 x^2 + (a - 2 \lambda m)x + b - m^2
\end{align*}

We also know this equation has roots at $x_1$, $x_2$ and $x_3$, and can be written as
\begin{align*}
    (x-x_1)(x-x_2)(x-x_3)=0 \\
    x^3 - (x_1 + x_2 + x_3)x^2 + (x_1x_2 + x_1x_3 + x_2x_3)x - x_1x_2x_3 = 0
\end{align*}

Comparing coefficients of $x^2$ in the two forms of the equation, we can see that \[\lambda^2 = x_1 + x_2 = x_3\] and therefore \[x_3 = \lambda^2 - x_1 - x_2\]

Which gives the equation for $x_3$. To find $y_3$, we use the fact that $\lambda$ is also equal to the gradient between $C'(x_3, -y_3)$ and $B$:
\begin{align*}
    \lambda = \frac{-y_3-y_1}{x_3-x_1} \\
    -y_3 - y_1 = \lambda(x_3-x_1) \\
    y_3 = \lambda(x_1-x_3)-y_1
\end{align*}
and we have the equation for $y_3$.

Adapting these equations for point doubling $A + A = C$ is easy; as the line is no longer through two points but rather the tangent at $A$, we differentiate $E$ to find the new gradient $\lambda$:
\begin{align*}
    y^2 = x^3 + ax + b \\
    2y\frac{dy}{dx} = 3x^2 + a \\
    \frac{dy}{dx} = \frac{3x^2 + a}{2y}
\end{align*}
and so \[\lambda = \frac{3x_1^2 + a}{2y_1}\]

Also, as $x_1 = x_2$, the equation for $x_3$ becomes \[x_3 = \lambda^2-2x_1\]

The equation for $y_3$ is unchanged.

\subsection{Point at infinity}

When the points of an elliptic curve are used as an abelian group $(G, +)$, we define an identity element, the \textit{point at infinity}, which is written as $\infty$, $\mathcal{O}$ or $0$. This is a point on the curve such that, for any point $A$:
\begin{enumerate}
    \item $A + \infty = A$
    \item $\infty + A = A$
    \item $A + A' = \infty$ where $A'$ is the inverse of $A$
    \item $\infty + \infty = \infty$
\end{enumerate}


\section{Elliptic curves over prime fields}

\subsection{Subgroups}


\section{Public key cryptography with elliptic curves}

\subsection{Point multiplication}

\subsection{Elliptic Curve Diffie-Hellman (ECDH)}

I am citing right now\cite{guide} and now\cite{discrete} and now\cite{practical}. And again\cite{guide}!

\newpage
\bibliographystyle{jox}
\bibliography{bibliography}

\end{document}