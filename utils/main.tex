\documentclass[a4paper]{article}

\usepackage[english]{babel}

\usepackage[a4paper,top=2cm,bottom=2cm,left=3cm,right=3cm,marginparwidth=1.75cm]{geometry}

\usepackage[%
  titleformat=italic,% Titles in italic 
  titleformat=commasep,% A comma between athors and title 
  titleformat=all,% Always show a title (or a short title)
  commabeforerest,% A comma after title 
  ibidem=strict,% 
  citefull=first,% The first citing in full form 
  oxford,% The oxford style
  super,% Footnotes 
  dotafter=true,% 
  see,% An extra optional argument as a prenote 
  idem
]{jurabib}

\usepackage{amsmath}
\usepackage{graphicx}
\usepackage[colorlinks=true, allcolors=blue]{hyperref}

\title{Elliptic curves in modern cryptography}
\author{Andrei Lupsa}

\begin{document}
\maketitle

\begin{abstract}
Abstract!!
\end{abstract}


\section{Introduction to elliptic curves}

An elliptic curve is an implicit function where one variable has a degree of 2 and the other has a degree of 3. 


\section{Elliptic curves as an abelian variety}

An abelian group is a kdfhjgkdfhjkg. A group $G$ that acts as an abelian group under an operation (such as addition, $+$) is written as $(G, +)$.

\subsection{Point addition \& doubling}

\subsection{Point at infinity}

When the points of an elliptic curve are used as an abelian group $(G, +)$, we need an identity point, the \emph{point at infinity}, which is written as $\infty$, $\mathcal{O}$ or $0$. This point is such that, for any point $A$:

\begin{enumerate}
    \item $A + \infty = A$
    \item $\infty + A = A$
    \item $A + -A = \infty$ where $-A$ is the inverse of $A$
    \item $\infty + \infty = \infty$
\end{enumerate}


\section{Elliptic curves over prime fields}


\section{Point multiplication}

\subsection{Hardness}


\section{Public key cryptography with elliptic curves}

I am citing right now\cite[3-4]{greenwade93} and even now\cite[5-6]{greenwade93}.

\newpage
\bibliographystyle{jox}
\bibliography{bibliography}

\end{document}